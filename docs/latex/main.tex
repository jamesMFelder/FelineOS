\documentclass[12pt]{article}
\usepackage[T1]{fontenc}
\usepackage{xcolor}
\definecolor{light-gray}{gray}{0.95}
\newcommand{\code}[1]{\colorbox{light-gray}{\texttt{#1}}}

\title{Feline OS Documentation}

\begin{document}

\begin{center}
\fontsize{25}{35}
\selectfont
\textbf{Feline OS Documentation}
\end{center}

\section{Overview}
\begin{list}{}{}
\item This is my hobby OS, don't expect anything professional.
\item I don't have a cross compiler with this because that would lock it to my type of machine.
\begin{tiny} The real reason is my distribution has one as a package.\end{tiny}

\end{list}

\section{Structure of the Code}
\begin{list}{}{}
\item \code{docs/} : Documentation
\item \code{docs/latex/} : Original latex documentation.
\item \code{docs/\{html,pdf\}/} : Generated documentation.
\item \code{kernel/} : The kernel.
\item \code{kernel/arch/} : Architecture dependent stuff in the kernel.
\item \code{kernel/arch/i386/} : 32-bit PC stuff
\item \code{kernel/arch/\$(ARCH)/} :
\item \code{kernel/arch/\$(ARCH)/boot/} : the early boot process
\item \code{kernel/arch/\$(ARCH)/gdt/} : dealing with the Global Descriptor Table \begin{small}(mostly creating the illusion of disabling it)\end{small}
\item \code{kernel/arch/\$(ARCH)/interrupts/} : handling interrupts
\item \code{kernel/arch/\$(ARCH)/screen/} : hardware output
\item \code{kernel/include/} : Kernel headers
\item \code{kernel/include/kernel/} : \code{<kernel/header.h>} instead of \code{<header.h>}
\item \code{kernel/kernel/} : Generic kernel files.
\item \code{libc/} : A minimal C library
\item \code{libc/arch/} : Architecture dependent libc stuff (macros, ...)
\item \code{libc/arch/i386/} : 32-bit PC stuff
\item \code{libc/include/} : Header files.
\item \code{libc/include/sys/} : System header files.
\item \code{libc/stdio/} : I/O programs.
\item \code{libc/stdlib/} : Other C library programs.
\item \code{libc/string/} : C String handling programs.
\item \code{utils/} : Programs to run outside the OS
\end{list}

\section{Documentation}
\begin{list}{}{}
\item \code{docs/latex} contains the latex documentation
\item Running \code{./docs.sh} creates documentation in \code{docs/\{pdf,html\}} if you have \code{pdflatex} and \code{latex2html} installed.
\item I try to comment in the code, but I often forget.
\end{list}

\section{Usefull Commands}
\begin{list}{}{}
\item \code{./default-host.sh} : Echo the architecture we are building for.
\item \code{./config.sh} : Define a bunch of variables.
\item \code{./headers.sh} : Copy headers into their correct places.
\item \code{./build.sh} : Compile FelineOS.
\item \code{./iso.sh} : Create an ISO.
\item \code{./qemu.sh} : Run the iso in qemu.
\item \code{./docs.sh} : Create the pdf and html documentation.
\item \code{./clean.sh} : Clean up everything.
\end{list}

\section{Usefull Programs}
\begin{list}{}{}
\item \code{utils/gdt\_create.c} : The main function calls \code{create\_descriptor(base, limit, type)}. The only defined types (currently) are \code{\{code,data\}-ring\{0,3\}}.
\end{list}

\end{document}
