% SPDX-License-Identifier: MIT
% Copyright (c) 2023 James McNaughton Felder

% Hexadecimal numbers (0x1) are not multiplying by zero (0×1)
% chktex-file 29

\documentclass[12pt]{article}
\usepackage[T1]{fontenc}
\usepackage{xcolor}
\definecolor{light-gray}{gray}{0.95}
\newcommand{\code}[1]{\colorbox{light-gray}{\texttt{#1}}}

\title{Feline OS Documentation}

\begin{document}

\begin{center}
\fontsize{25}{35}
\selectfont
\textbf{Feline OS Documentation}
\end{center}

\section{Overview}
\begin{list}{}{}
\item This is my hobby OS, don't expect anything professional.
\item The kernel is written in C++ so the kernel and terminal library headers are \textbf{not} C compatible, but the libc and libFeline headers should be.
\end{list}

\section{Structure of the Code}
\begin{list}{}{}
\item \code{docs/}: Documentation
\item \code{docs/latex/}: Original latex documentation.
\item \code{docs/\{html,pdf\}/}: Generated documentation.
\item \code{system/kernel/}: The kernel.
\item \code{system/kernel/arch/}: Architecture dependent stuff in the kernel.
\item \code{system/kernel/arch/i386/}: 32-bit PC stuff
\item \code{system/kernel/arch/arm/}: ARM (specifically Raspberry Pi) stuff
\item \code{system/kernel/arch/\$(ARCH)/}:
\item \code{system/kernel/arch/\$(ARCH)/include}: Architecture-specific headers (mostly for drivers)
\item \code{system/kernel/arch/\$(ARCH)/boot/}: the early boot process
\item \code{system/kernel/arch/\$(ARCH)/gdt/}: dealing with the Global Descriptor Table \begin{small}(mostly creating the illusion of disabling it)\end{small}
\item \code{system/kernel/arch/\$(ARCH)/interrupts/}: handling interrupts
\item \code{system/kernel/arch/\$(ARCH)/mem/}: bootstrapping physical memory
\item \code{system/kernel/arch/\$(ARCH)/stack/}: backtraces
\item \code{system/kernel/arch/\$(ARCH)/task/}: low-level switching between tasks (processes, threads)
\item \code{system/kernel/arch/\$(ARCH)/virt\_mem/}: managing paging
\item \code{system/kernel/drivers/}: Stuff that one might or might not want
\item \code{system/kernel/drivers/include/}: Header files for the drivers
\item \code{system/kernel/drivers/include/drivers/}: \code{<drivers/header.h>} instead of \code{<header.h>}
\item \code{system/kernel/drivers/framebuffer/}: Code for a framebuffer on x86.
\item \code{system/kernel/drivers/PIT/}: Code for the x86 Programmable Interrupt Timer.
\item \code{system/kernel/drivers/pl011/}: Code for UART-like pl011 on (at least) the Raspberry Pi.
\item \code{system/kernel/drivers/systimer/}: Code for ARM system timer.
\item \code{system/kernel/drivers/uart/}: Code for the UART serial port on x86.
\item \code{system/kernel/drivers/vga/}: Code for the x86 text-based VGA
\item \code{system/kernel/include/}: Kernel headers.
\item \code{system/kernel/include/kernel/}: \code{<kernel/header.h>} instead of \code{<header.h>}
\item \code{system/kernel/kernel/}: Generic kernel files.
\item \code{system/libc/}: A minimal C library
\item \code{system/libc/arch/}: Architecture dependent libc stuff (system calls, …)
\item \code{system/libc/arch/\$(ARCH)/syscall/}: system call functions
\item \code{system/libc/assert/}: Internal functions for the assert macro.
\item \code{system/libc/ctype/}: Character types
\item \code{system/libc/errno/errno.cpp}: A c++ file to hold the declaration of errno.
\item \code{system/libc/fcntl/}: open and close functions
\item \code{system/libc/icxxabi/}: atexit ABI stuff that can't be in a header.
\item \code{system/libc/include/}: Header files.
\item \code{system/libc/include/bits/}: Internals
\item \code{system/libc/include/c++/}: C++ header files
\item \code{system/libc/include/sys/}: System header files.
\item \code{system/libc/ssp/}: What gcc needs to detect stack smashing.
\item \code{system/libc/stdio/}: I/O programs.
\item \code{system/libc/stdlib/}: Other C library programs.
\item \code{system/libc/string/}: C String handling programs.
\item \code{system/libc/unistd/}: read and write functions
\item \code{system/libFeline/}: Functions and headers that aren't part of the C standard or the kernel.
\item \code{system/libFeline/include/}: Header files (which is most of the code, due to how many templates there are)
\item \code{system/libFeline/include/feline}: \code{<feline/header.h>} instead of \code{header.h}
\item \code{system/libFeline/src}: source files
\item \code{system/libFeline/src/allocator}: tiny wrappers around malloc and free for KGeneralAllocator
\item \code{system/libFeline/src/locking}: spinlock code
\item \code{system/libFeline/src/logging}: kout class, and supporting functions
\item \code{system/libFeline/src/settings}: a c++ file to instantiate the settings objects
\item \code{system/libFeline/src/strings}: utilities for formatting numbers as strings
\item \code{system/libFeline/src/vector}: an error function for out-of-bounds access, to prevent recursive inclusion of kout
\item \code{system/libFeline/tests}: Tests for most of the classes
\item \code{system/terminals/}: Terminal classes.
\item \code{system/terminals/common}: A base class everything else comes from.
\item \code{system/vga/}: VGA terminals (only text-mode supported now).
\item \code{system/include/terminals/}: \code{<terminals/header.h>} instead of \code{<header.h>}
\item \code{system/include/terminals/vga}: VGA terminal header files.
\item \code{utils/}: Programs to run outside the OS
\end{list}

\section{Documentation}
\begin{list}{}{}
\item \code{docs/latex} contains the latex documentation
\item Running \code{./docs.sh} creates documentation in \code{docs/\{pdf,html\}} if you have \code{pdflatex} and \code{latex2html} installed.
\item \code{docs/Architecture.{md,html} has more overview-focused documentation}
\item I try to comment in the code, but I often forget.
\end{list}

\section{Useful Programs}
\begin{list}{}{}
\item \code{utils/gdt\_create.cpp}: The main function calls \code{create\_descriptor(base, limit, type)}. The only defined types (currently) are \code{\{code,data\}-ring\{0,3\}}.
\item \code{utils/sizes.cpp}: Print the sizes of various types. Use \code{-{}-json} if wanted.
\item \code{utils/sizes\_freestanding.cpp}: Print the sizes of various types. Only uses freestanding available types. Try \code{\$target-compiler -S} to get the assembly without linking if needed.
\end{list}

\section{Memory management}
\begin{list}{}{}
\item The PMM has a linked-list that currently only tracks used/not used.
\item Reserve memory with \code{pmm\_results get\_mem\_area(void **addr, uintptr\_t len)} or \code{pmm\_results get\_mem\_area(void const * const addr, uintptr\_t len);}
\item Return it with \code{free\_mem\_area(void const * const addr, uintptr\_t len)}
\item The VMM keeps track of paging. It is currently only able to handle the main process.
\item Map pages with \code{map\_range()} (various types of arguments)
\item Unmap pages with \code{unmap\_range()}.
\end{list}

\section{IO}
\begin{list}{}{}
\item libFeline has kout (a qDebug-style class), which use use with kDbg, kLog, kWarn, kCritical…
\begin{list}{}{}
\item Output numbers with the functions dec(), hex, ptr().
\item This is generally the most convenient way, but it does dynamically allocate memory (to support future non-interleaving output).
\item This is technically avoidable with the *NoDebug variants, but the number-formatting functions still allocate memory anyways, so unless you have a constant string, printf/klogf is a better bet.
\end{list}
\item Currently all standard output functions (\code{printf()}, \code{puts()}, \code{putchar()}, \ldots) all output to the serial port (\code{serial\_writestring()}, \code{serial\_write()}, \code{serial\_putchar()}).\linebreak
\begin{list}{}{}
\item Please only use standard output functions (\code{printf}, \code{puts}, \code{putchar}, \ldots) or libFeline functions. I make no guarantees about supporting any other functions.
\end{list}
\item There is no input (yet).
\end{list}

\section{Tasks}
This does not cover user-space anything yet!
\begin{list}{}{}
\item \code{add\_new\_task} takes as an argument a function (or lambda) to execute that will never return
\item \code{sched} \textit{can} be called manually, however it might just return instantly if no other processes are ready to run, so it's most useful for pre-emption.
\item \code{end\_cur\_task} just marks the current task as finished, and attempts to switch to a different one. Note that if there is no other task, it currently makes the current task a permanent idle task, but that is considered a bug. Also, the actual resource reclamation (including allocations such as the stack) are saved for later, by a different process.
\item \code{cleanup\_finished\_tasks} goes over every completed task and does the actual freeing of resources. This is important to avoid freeing the stack currently in use!
\end{list}
Note that currently there is no process separation: everything is in one global address space.

\section{Syscalls}
\begin{list}{}{}
\item Call using \code{int 31}/\code{int 0x1F} (NASM) or \code{int \$31}/\code{int \$0x1f} (GAS) for x86
\item On ARM, use \code{svc \#0}
\item The syscall number is in \code{eax}/\code{r0}.
\item The pointer to the argument structure is in \code{ebx}/\code{r1}.
\item The pointer to the result structure is in \code{ecx}/\code{r2}.
\item C(++) code should use the \code{raw\_syscall} wrapper function in \code{<sys/syscall.h>}, if it can't use the wrapper functions.
\item See Architecture.md for more details on how the mechanism works.
\end{list}
\end{document}
