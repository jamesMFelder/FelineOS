% SPDX-License-Identifier: MIT
% Copyright (c) 2023 James McNaughton Felder

% Hexadecimal numbers (0x1) are not multiplying by zero (0×1)
% chktex-file 29

\documentclass[12pt]{article}
\usepackage[T1]{fontenc}
\usepackage{xcolor}
\definecolor{light-gray}{gray}{0.95}
\newcommand{\code}[1]{\colorbox{light-gray}{\texttt{#1}}}

\title{Feline OS Documentation}

\begin{document}

\begin{center}
\fontsize{25}{35}
\selectfont
\textbf{Feline OS Documentation}
\end{center}

\section{Overview}
\begin{list}{}{}
\item This is my hobby OS, don't expect anything professional.
\item The kernel is written in C++ so the kernel and terminal library headers are \textbf{not} C compatible, but the libc and libFeline headers should be.
\end{list}

\section{Structure of the Code}
\begin{list}{}{}
\item \code{docs/}: Documentation
\item \code{docs/latex/}: Original latex documentation.
\item \code{docs/\{html,pdf\}/}: Generated documentation.
\item \code{system/kernel/}: The kernel.
\item \code{system/kernel/arch/}: Architecture dependent stuff in the kernel.
\item \code{system/kernel/arch/i386/}: 32-bit PC stuff
\item \code{system/kernel/arch/\$(ARCH)/}:
\item \code{system/kernel/arch/\$(ARCH)/boot/}: the early boot process
\item \code{system/kernel/arch/\$(ARCH)/gdt/}: dealing with the Global Descriptor Table \begin{small}(mostly creating the illusion of disabling it)\end{small}
\item \code{system/kernel/arch/\$(ARCH)/interrupts/}: handling interrupts
\item \code{system/kernel/arch/\$(ARCH)/mem/}: managing physical memory
\item \code{system/kernel/drivers/}: Stuff that one might or might not want
\item \code{system/kernel/drivers/include/}: Header files for the drivers
\item \code{system/kernel/drivers/include/drivers/}: \code{<drivers/header.h>} instead of \code{<header.h>}
\item \code{system/kernel/drivers/serial/}: Code for the serial port.
\item \code{system/kernel/include/}: Kernel headers.
\item \code{system/kernel/include/kernel/}: \code{<kernel/header.h>} instead of \code{<header.h>}
\item \code{system/kernel/kernel/}: Generic kernel files.
\item \code{system/libc/}: A minimal C library
\item \code{system/libc/arch/}: Architecture dependent libc stuff (macros, …)
\item \code{system/libc/arch/i386/}: 32-bit PC stuff
\item \code{system/libc/arch/\$(ARCH)/syscall/}: system call functions
\item \code{system/libc/ctype/}: Character types
\item \code{system/libc/include/}: Header files.
\item \code{system/libc/include/bits/}: Internals
\item \code{system/libc/include/c++/}: C++ header files
\item \code{system/libc/include/sys/}: System header files.
\item \code{system/libc/ctype/}: Detecting character types.
\item \code{system/libc/ssp/}: What gcc needs to detect stack smashing.
\item \code{system/libc/stdio/}: I/O programs.
\item \code{system/libc/stdlib/}: Other C library programs.
\item \code{system/libc/string/}: C String handling programs.
\item \code{system/libFeline/}: Functions and headers that aren't part of the C standard or the kernel.
\item \code{system/libFeline/arch}: Architecture-dependent functions
\item \code{system/libFeline/arch/i386}: 32-bit PC functions
\item \code{system/libFeline/include/feline}: \code{<feline/header.h>} instead of \code{header.h}
\item \code{system/libFeline/include/feline/asm}: Assembly headers (NASM ones end in \code{.inc}, haven't decided for GAS)
\item \code{system/terminals/}: Terminal classes.
\item \code{system/terminals/common}: A base class everything else comes from.
\item \code{system/vga/}: VGA terminals (only text-mode supported now).
\item \code{system/include/terminals/}: \code{<terminals/header.h>} instead of \code{<header.h>}
\item \code{system/include/terminals/vga}: VGA terminal header files.
\item \code{utils/}: Programs to run outside the OS
\end{list}

\section{Documentation}
\begin{list}{}{}
\item \code{docs/latex} contains the latex documentation
\item Running \code{./docs.sh} creates documentation in \code{docs/\{pdf,html\}} if you have \code{pdflatex} and \code{latex2html} installed.
\item I try to comment in the code, but I often forget.
\end{list}

\section{Usefull Commands}
\begin{list}{}{}
\item \code{./default-host.sh}: Echo the architecture we are building for.
\item \code{./config.sh}: Define a bunch of variables.
\item \code{./headers.sh}: Copy headers into their correct places.
\item \code{./build.sh}: Compile FelineOS.\@
\item \code{./iso.sh}: Create an iso.
\item \code{./qemu.sh}: Run the iso in qemu.
\item \code{./docs.sh}: Create the pdf and html documentation.
\item \code{./clean.sh}: Clean up everything.
\end{list}

\section{Usefull Programs}
\begin{list}{}{}
\item \code{utils/gdt\_create.cpp}: The main function calls \code{create\_descriptor(base, limit, type)}. The only defined types (currently) are \code{\{code,data\}-ring\{0,3\}}.
\item \code{utils/sizes.cpp}: Print the sizes of various types. Use \code{-{}-json} if wanted.
\item \code{utils/sizes\_freestanding.cpp}: Print the sizes of various types. Only uses freestanding available types. Try \code{\$target-compiler -S} to get the assembly without linking if needed.
\end{list}

\section{Memory management}
\begin{list}{}{}
\item The PMM has a bitmap that currently only tracks used/not used.
\item Reserve memory with \code{pmm\_results get\_mem\_area(void **addr, uintptr\_t len, unsigned int opts)} or \code{pmm\_results get\_mem\_area(void const * const addr, uintptr\_t len, unsigned int opts);}
\item Return it with \code{free\_mem\_area(void const * const addr, uintptr\_t len, unsigned int opts)}
\item Note: opts does nothing, but is here for future-proofness.
\item The VMM keeps track of paging. It is currently only able to handle the main process.
\item Map pages with \code{map\_range()} (various types of arguments)
\item Unmap pages with \code{unmap\_range()}.
\end{list}

\section{IO}
\begin{list}{}{}
\item Currently all standard output functions (\code{printf()}, \code{puts()}, \code{putchar()}, \ldots) all output to the serial port (\code{serial\_writestring()}, \code{serial\_write()}, \code{serial\_putchar()}).\linebreak
\begin{list}{}{}
\item Please only use standard output functions (\code{printf}, \code{puts}, \code{putchar}, \ldots). I make no garantees about supporting any other functions.
\end{list}
\item There is no input (yet).
\end{list}

\section{Syscalls}
\begin{list}{}{}
\item Call using \code{int 31}/\code{int 0x1F} (NASM) or \code{int \$31}/\code{int \$0x1f} (GAS)
\item The syscall number is in \code{eax}.
\item The arguments are in \code{ecx} and \code{edx} (any further are pushed onto the stack)
\item If they don't fit in a register (\code{32} bits on \code{i686}), they are pointers.
\item C(++) code can use the \code{syscall} wrapper function in \code{<sys/syscall.h>}
\item Currently return \code{-1} because none are supported.
\end{list}
\end{document}
